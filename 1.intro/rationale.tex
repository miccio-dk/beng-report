\section{Rationale}

From map-making to post tracking, localization has been a fundamental human endeavor which encompassed several ages and civilizations.
Although GPS has proven to be a major leap is this regard, its current limitations causes it to be unsuitable for some applications.
First of all, the accuracy lays in the order of a few meters at best, mainly due to the relativistic implication of orbiting satellites \cite{gps1} (that is without considering more advanced techniques such as \emph{RTK}).
Moreover, signal shadowing and multi-path reflections renders it particularly unreliable indoor or in dense urban areas.
In order to fulfill the ever-growing demand for automation, more precise location systems have to be employed.

The term RTLS emerged from such needs, and characterizes systems capable of reckoning an tracking the position of moving objects and persons within a relative frame of reference, mainly a building or enclosed area.
The technologies driving such systems include blabla, although each of them presents its own limitations which prevented them to be used
Ultra-wide blabla is a relatively new entry in the RTLS blabla


Explanation on why RTLS and UAVCAN are useful.
