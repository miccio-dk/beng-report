% !TEX root = ../root.tex
\section{Requirements specification}\label{sec:reqs}
The \emph{software requirements specification} is the formal description of the system to be developed, thoroughly describing what the product is --- and is not --- expected to do.
It also helps assessing the extent of the endeavor in terms of workload, cost, and other resources.
The requirements are commonly laid out in agreement with the client, and provide the basis upon which the product or project is evaluated.

As far as this project is concerned, the company has not set any particular requirements apart from those implied in the project name.
The list of requirements is therefore mostly based on former knowledge, common sense and best practices within the industry,


\subsection{Functional requirements}\label{subsec:req_func}
This section defines specific behaviors or functionalities, and is the basis for the tests.
Requirements introduced by the modal verb \emph{shall} indicate mandatory conditions, whilst those introduced by the modal verb \emph{may} indicate optional features.

For conciseness and clarity sake, along with tag and anchor, the concept of \emph{device}is also introduced.
Requirements specified for the device are to be fulfilled by both elements.

\begin{table}[H]
\centerfloat
\begin{tabular}{@{} m{6em} >{\small}l @{}}
    \toprule
    ID      & \normalfont{Description} \\
    \midrule
    SRS.F.HW.1    & The tag shall include a UAVCAN-compliant interface \\
    SRS.F.HW.2    & The tag shall be able to be powered through CAN \\
    SRS.F.HW.3    & The device shall be able to be powered through USB \\
    SRS.F.HW.4    & The tag shall provide visual feedback for errors and critical states \\
    SRS.F.HW.5    & The tag shall provide visual feedback for CAN activity \\
    SRS.F.HW.5    & The tag shall be able to fit inside a typical copter drone \\
    \midrule
    SRS.F.SW.1    & The tag shall be able to notify its presence and status in the UAVCAN bus \\
    SRS.F.SW.2    & The tag shall be able to obtain a node ID from a dynamic node ID allocator server \\
    SRS.F.SW.3    & The tag shall be able to respond to node discovery requests \\
    SRS.F.SW.4    & The tag shall be able to reboot when requested by another node \\
    SRS.F.SW.5    & The tag shall be able to synchronize its time with a UAVCAN master clock node \\
    SRS.F.SW.6    & The tag shall be able to emit debugging information and logs, if requested \\
    SRS.F.SW.7    & The tag shall be able to be configured through UAVCAN \\
    SRS.F.SW.8    & The tag shall be able to discover anchors in the proximity \\
    SRS.F.SW.9    & The tag shall be able to periodically poll the anchors for new ranging samples \\
    SRS.F.SW.10   & The tag shall be able to update its location estimation when a new sample is available \\
    SRS.F.SW.11   & The tag shall be able to publish its location data through the UAVCAN bus \\
    SRS.F.SW.12   & The anchors shall be able to respond to tag discovery request \\
    SRS.F.SW.13   & The anchors shall be able to respond to tag ranging requests \\
    SRS.F.SW.14   & The device shall be able to be configured through USB \\
    SRS.F.SW.15   & The tag may be able to update its firmware through the UAVCAN bus \\
    SRS.F.SW.16   & The tag may use the onboard sensors to refine the location data \\
    SRS.F.SW.17   & The device may implement IEEE 802.15.4 frame format for communication \\
    \bottomrule
\end{tabular}
\caption{Software requirements specification: functional requirements}\label{tab:srs_fun}
\end{table}


\subsection{Non functional requirements}\label{subsec:req_nf}
This section specifies desired characteristics and qualities of the system.

\begin{table}[H]
\centerfloat
\begin{tabular}{@{} m{6em} >{\small}l @{}}
    \toprule
    ID      & \normalfont{Description} \\
    \midrule
    SRS.NF.SW.1    & The source shall be written in C++ \\
    SRS.NF.SW.2    & The source shall strive towards modularity \\
    SRS.NF.SW.3    & The source shall not allocate or make us heap-based memory \\
    SRS.NF.SW.4    & The source shall follow a consistent format style \\
    SRS.NF.SW.5    & The source shall be properly documented \\
    SRS.NF.SW.6    & The released documentation shall be written in English \\
    SRS.NF.SW.7    & The source may be supported by other toolchains and compilers  \\
    \bottomrule
\end{tabular}
\caption{Software requirements specification: non-functional requirements}\label{tab:srs_nfun}
\end{table}


\subsection{Use cases}\label{subsec:u_cases}
This section describes how the actors can interact with the system.
Since the product is meant to be used in an unmanned and autonomous environment, the definition of actor becomes quite loose.
The identified actors are the AUV autopilot and the final user.

As most of the identified use case are tightly couple with a functional requirements, they have been grouped and cross-referenced.

\fig{10cm}{use_case.png}{Software requirements specification: use cases}{use_case}
