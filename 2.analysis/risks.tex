% !TEX root = ../root.tex
\section{Risk management}

Risk assessment and management are well-established disciplines amongst many fields of engineering and other areas of human endeavor, and represent a way of minimizing the probability and impact of adverse conditions in a given context.

In the \nameref{subsec:overview} subsection, possible sources of hindrance are categorized and listed.
Preventive measures are then suggested in the \nameref{subsec:prevention} subsection.


\subsection{Description}
This section acts as a legend, providing the necessary information for interpreting the subsequent risk assessment.
Risks are evaluated according to their composite risk index (\emph{CRI}):
\begin{equation}
    \text{CRI} = \text{Severity} \times \text{Probability of occurrence}
\end{equation}
The CRI represents the degree of importance of each entry.
Entries with a high (ie. red-leaning) CRI shall have maximum priority.

Traditionally, this kind of analysis involves assigning a numeric value from an arbitrarily chosen range to each variable --- e.g. 1--5 for the severity and 0--100\% for the probability.
However, educated estimations require solid data and a tracked history of previous occurrences, all of which were either not available or difficult to obtain with the necessary degree of confidence. \cite{risk}
Therefore, the final value of the CRI has been neglected in favor of a color-mapped matrix composed of four sub-ranges.
Whilst selecting these sub-ranges, higher ``granularity'' has been given to the more adverse scenarios.

\begin{table}[!h]
\centering
\newcolumntype{P}[1]{>{\centering\arraybackslash}p{#1}}
\begin{tabular}{c r P{5em} P{5em} P{5em} P{5em}}
    \toprule
    &               & \multicolumn{4}{c}{\emph{Probability of occurrence}}                                  \\
    \cmidrule{3-6}
    &               & Negligible          & Low                & Medium              & High                \\
    &               & (1)                 & (2)                & (3)                 & (4)                 \\
    \multirow{4}{*}{\rotatebox[origin=c]{90}{\emph{Severity}}}
    & Small (1)     & \cellcolor{Green}   & \cellcolor{Green}  & \cellcolor{Yellow}  & \cellcolor{Yellow}  \\
    & Moderate (2)  & \cellcolor{Green}   & \cellcolor{Yellow} & \cellcolor{Yellow}  & \cellcolor{Orange}  \\
    & Large (3)     & \cellcolor{Yellow}  & \cellcolor{Yellow} & \cellcolor{Orange}  & \cellcolor{Red}     \\
    & Critical (4)  & \cellcolor{Yellow}  & \cellcolor{Orange} & \cellcolor{Red}     & \cellcolor{Red}     \\
    \bottomrule
\end{tabular}
\caption{Risk impact matrix}\label{tab:risk_legend}
\end{table}


\subsection{Overview}\label{subsec:overview}
Follows the lists of threats (divided by category) with associated likelihood, impact, and overall CRI.
The risk index is accompanied with the affected resource(s), chosen amongst:
\begin{itemize}
    \item Product \textbf{F}unctionality
    \item Project \textbf{C}ost
    \item Development \textbf{T}ime
\end{itemize}
Whenever possible or necessary, a brief explanation of the predicted consequences is given.

\subsubsection{Hardware}
The estimations are based on company personnel's knowledge and previous experience with the given components and communication protocols governing them.
For the purpose in this analysis the board components have been divided in primary (microcontroller, DWM1000 module, CAN transciever) and secondary (power supply unit, other sensors, USB interface), based on their importance for the fulfillment of the project.
\begin{table}[!h]
\centering
\newcommand{\coldot}[1][black]{\Large\textcolor{#1}{\ensuremath\bullet}}
\newcolumntype{P}[1]{>{\centering\arraybackslash}p{#1}}
\begin{tabular}{@{} l l P{2em} P{2em} P{2em} P{2em} >{\small}l @{}}
    \toprule
    ID      & Risk                    & Prob. & Sev. & CRI              & Res. & \normalfont{Contingencies/Notes} \\
    \midrule
    SD      & \multicolumn{6}{l}{\emph{Schematics design issues (wrong/missing parts, connections\dots)}} \\
    SD.1    & Primary components      & 3     & 4    & \coldot[Red]     & TC   & New PCB manufacturing \\
    SD.2    & Secondary components    & 2     & 3    & \coldot[Yellow]  & TF   & \\
    \addlinespace
    PL      & \multicolumn{6}{l}{\emph{PCB layout issues (wrong footprints, overlapping traces\dots)}} \\
    PL.1    & Primary components      & 3     & 3    & \coldot[Orange]  & TC   & New PCB manufacturing \\
    PL.2    & Secondary components    & 1     & 2    & \coldot[Green]   & TF   & \\
    \addlinespace
    SC      & \multicolumn{6}{l}{\emph{Supply chain issues}} \\
    SC.1    & Components lead time    & 1     & 3    & \coldot[Yellow]  & T    & Wait until restocked \\
    SC.2    & PCB lead time           & 3     & 3    & \coldot[Orange]  & T    & Negotiation/bargaining \\
    SC.3    & Defective PCB           & 1     & 4    & \coldot[Yellow]  & T    & New PCB delivery \\
    \addlinespace
    PU      & \multicolumn{6}{l}{\emph{Performance/usability issues}} \\
    PU.1    & MCU doesn't boot up     & 1     & 4    & \coldot[Yellow]  & T    & \\
    PU.2    & MCU not detected by SWD & 2     & 4    & \coldot[Orange]  & T    & Tedious troubleshooting \\
    PU.3    & High packet loss (USB, CAN) & 1 & 2    & \coldot[Green]   & F    & \\
    PU.4    & DWM1000 calibration off & 2     & 2    & \coldot[Yellow]  & F    & \\
    \bottomrule
\end{tabular}
\caption{Risk assessment: Hardware-related threats}\label{tab:risk_hw}
\end{table}

\subsubsection{Software}
The estimations are based on personal previous development experience, online research, and relevant documentation.
The table sub-categories loosely reflect the different software layers described in the \nameref{sec:software} section.
\begin{table}[!h]
\centering
\newcommand{\coldot}[1][black]{\Large\textcolor{#1}{\ensuremath\bullet}}
\newcolumntype{P}[1]{>{\centering\arraybackslash}p{#1}}
\begin{tabular}{@{} l l P{2em} P{2em} P{2em} P{2em} >{\small}l @{}}
    \toprule
    ID      & Risk                    & Prob. & Sev. & CRI              & Res. & \normalfont{Contingencies/Notes} \\
    \midrule
    DE      & \multicolumn{6}{l}{\emph{Development environment issues}} \\
    DE.1    & No compiler available   & 1     & 4    & \coldot[Yellow]  & C    & Switch MCU vendor/arch. or host system \\
    DE.2    & No flashing tools available & 2 & 4    & \coldot[Orange]  & C    & Switch MCU vendor/arch. or host system \\
    DE.3    & Loss of data            & 2     & 3    & \coldot[Yellow]  & T    & \\
    \addlinespace
    LL      & \multicolumn{6}{l}{\emph{Low-level support issues (peripherals, data busses\dots)}} \\
    LL.1    & No MCU peripherals drivers & 2  & 3    & \coldot[Yellow]  & TF   & Implement from scratch \\
    LL.2    & No MCU peripherals docs. & 1    & 4    & \coldot[Yellow]  & TC   & Switch MCU vendor/arch. \\
    LL.2    & No comm. with ext. devices & 2  & 4    & \coldot[Orange]  & TF   & \\
    LL.4    & RTOS not supported      & 2     & 1    & \coldot[Green]   & TF   & Implement port; single thread paradigm \\
    \addlinespace
    MW      & \multicolumn{6}{l}{\emph{Middleware issues (libraries, device APIs\dots)}} \\
    MW.1    & Restrictive/costly license & 1  & 3    & \coldot[Yellow]  & CF   & Purchase; Implement from scratch \\
    MW.2    & Lack of platform support & 3    & 2    & \coldot[Yellow]  & TF   & Implement port \\
    MW.3    & Poor documentation      & 2     & 3    & \coldot[Yellow]  & TF   & \\
    MW.4    & Unstable/buggy code     & 1     & 2    & \coldot[Green]   & F    & \\
    \addlinespace
    UA      & \multicolumn{6}{l}{\emph{User application-level issues}} \\
    UA.1    & Reached computational limit & 1 & 3    & \coldot[Yellow]  & F    & \\
    UA.2    & Middlewares conflict    & 2     & 3    & \coldot[Yellow]  & TF   & \\
    UA.3    & Poor ranging accuracy   & 2     & 3    & \coldot[Yellow]  & F    & \\
    UA.5    & Unimplemented use cases & 2     & 4    & \coldot[Orange]  & T    & Increased time-to-market \\
    UA.6    & Unmet requirements      & 2     & 4    & \coldot[Orange]  & T    & Increased time-to-market \\
    \bottomrule
\end{tabular}
\caption{Risk assessment: Software-related threats}\label{tab:risk_sw}
\end{table}


\subsection{Prevention and mitigation}\label{subsec:prevention}

After having discerned and evaluated as many threats as possible, a number of preemptive measures will be suggested.
Four approaces blablabla

Due to the similarities amongst related entries, the risk prevention methods will be discussed on a sub-category basis.

\begin{table}[!h]
\centering
\newcommand{\coldot}[1][black]{\Large\textcolor{#1}{\ensuremath\bullet}}
\begin{tabular}{@{} l l >{\small}p{5em} l @{}}
    \toprule
    ID & \phantom{M} & \normalfont{Description} & Approach \\
    \midrule
    DE      & \multicolumn{3}{l}{\emph{Development environment issues}} \\
    \addlinespace
    LL      & \multicolumn{3}{l}{\emph{Low-level support issues (peripherals, data busses\dots)}} \\
    \addlinespace
    MW      & \multicolumn{3}{l}{\emph{Middleware issues (libraries, device APIs\dots)}} \\
    \addlinespace
    UA      & \multicolumn{3}{l}{\emph{User application-level issues}} \\
    \bottomrule
\end{tabular}
\caption{Risk management: threats preventions and mitigation proposal}\label{tab:risk_mgmt}
\end{table}
