% !TEX root = ../root.tex
\section{Theoretical principles}\label{sec:theory}
This section will present the theoretical aspects of the technologies employed in the project, and provides a knowledge base for better understanding the problem and the employed solution.


\subsection{Ultra-wideband}
Ultra-wideband (UWB in short) is a radio technology that use a very low energy level for short-range, high-bandwidth communications over a large portion of the radio spectrum.

% http://www.informit.com/articles/article.aspx?p=433381&seqNum=2

Although this technique has been know for for decades, is a relatively new entry in the RTLS field \cite{uwb1}, with the first affordable development kits having entered the market in the last few years.
It gained immediate popularity due to its advantages: blablabla


\subsection{Ranging algorithm}
There can be several possible two-way ranging schemes, blabla
% http://bespoon.com/two-way-ranging/
% https://en.wikipedia.org/wiki/Symmetrical_double-sided_two-way_ranging

Their role is to respond to the polling message sent by the tags after predetermined delay. This response message constains the polling message reception timestamp and its own transmission timestamp.

\subsection{Position estimation}
Explanation of the triangulation algorithms.

\subsection{CAN Bus}
Brief explanation of CAN protocol, its history, and principles of operation.


Most CAN interfaces support automatic filtering of undesired frames through acceptance filter.
The process consists in performing a bitwise \emph{and} operation between the incoming frame ID and the acceptance mask, and then comparing the result with the content of a filter register: if these match, the frame is added to the reception queue.
In this way, a receiving node can choose to accept frames of a specific type, or sent by a particular other node.
