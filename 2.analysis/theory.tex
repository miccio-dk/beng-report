\section{Theoretical principles}\label{sec:theory}
This section will present the theoretical aspects of the technologies employed in the project, and provides a knowledge base for better understanding the problem and the employed solution.


\subsection{Ultra-wideband}
Ultra-wideband (UWB in short) is a radio technology that use a very low energy level for short-range, high-bandwidth communications over a large portion of the radio spectrum.

http://www.informit.com/articles/article.aspx?p=433381&seqNum=2

Although this technique has been know for for decades, is a relatively new entry in the RTLS field \cite{uwb1}, with the first affordable development kits having entered the market in the last few years.
It gained immediate popularity due to its advantages: blablabla


\subsection{Ranging algorithm}
There can be several possible two-way ranging schemes, blabla
http://bespoon.com/two-way-ranging/
https://en.wikipedia.org/wiki/Symmetrical_double-sided_two-way_ranging

Their role is to respond to the polling message sent by the tags after predetermined delay. This response message constains the polling message reception timestamp and its own transmission timestamp.

\subsection{Position estimation}
Explanation of the triangulation algorithms.

\subsection{CAN Bus}
Brief explanation of CAN protocol, its history, and principles of operation.
May also be opted out and substituted with notions of ARM Cortex-M architecture.
