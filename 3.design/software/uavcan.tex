% !TEX root = ../../root.tex
\subsection{UAVCAN}\label{subsec:uavcan}
UAVCAN is a lightweight protocol aimed to provide fast and reliable communication within a drone peripherals.

A UAVCAN network is composed of a series of nodes, each one identified by a unique ID.
Each node is independent from the others and require no master/controller node or host computer.
Communication between the nodes is called \emph{transfer} and can occur through one of these two paradigms: message broadcasting (publisher-subscriber logic) or service invocation (request-response logic).

A CAN frame is composed of an ID and a payload.
The payload section contains the encoded message data only, so all the relevant meta-data (source node, message type, destination node in case of service request, etc\dots) must figure in the frame ID.

\fig{10cm}{uc_frm_id.png}{UAVCAN protocol: Frame ID formats}{uc_frm_id}

\autoref{fig:uc_frm_id} shows the arrangement of the frame ID for different transfer types.
Although UAVCAN frame IDs are 29 bits in size (\emph{CAN 2.0B}), its bus can be shared with protocols using the former 11 bits format (\emph{CAN 2.0A}).
\emph{Anonymous message frames} are those sent by nodes which do not have a default node ID and have not been assigned one yet.
In order to avoid CAN collision deadlock, the discriminator section has to be filled with random data.

\fig{10cm}{uc_frm_pl.png}{UAVCAN protocol: Payload format}{uc_frm_pl}

The payload structure is the same regardless of transfer type.
In single-frame transfers, the \emph{start of transfer} and \emph{end of transfer} bits shown in \autoref{fig:uc_frm_pl} are both set to one.

Since each transfer may contain up to eight bytes of data per frame, multi-frame transfers are also available.
In this case, the first two bytes of the first frame will contain the CRC of the complete payload, and the bit-flags in the tail byte are to be set accordingly.
The \emph{toggle} bit is set to 0 for even frames and 1 for odd ones.
The \emph{transfer ID} bits contain an incremental transfer counter, and its value must be the same for each frame of a given transfer.

Messages and services can be standard or vendor-specific.
The standard ones are divided in \emph{protocol} and \emph{equipment}, with the former enabling UAVCAN-related features and the latter providing data definitions for commonly used devices like navigation systems, airspeed sensors, and so forth.

Compliant node are only required to broadcast a heartbeat message on a regular basis to signal their presence.
The implementation of other standard functionalities is not mandatory but recommended.
For instance, nodes that do not implement \texttt{uavcan.protocol.GlobalTimeSync}-enabled time synchronization are condemned to experience time drift.
Here follows a table of UAVCAN high-level functions along with the messages/service through it is carried out, and --- if available --- the use case(s) or requirement(s) it fulfills.

\begin{table}[H]
\centerfloat
\begin{tabular}{@{} m{10em} >{\ttfamily}m{15em} m{5em} >{\small}m{10em} @{}}
    \toprule
    Function                   & \normalfont{Service/Message definition} & Applies to & \normalfont{Notes} \\
    \midrule
    Node status reporting      & uavcan.protocol.NodeStatus &  & Mandatory \\
    Node discovery             & uavcan.protocol.GetNodeInfo &  & Strongly suggested \\
    Time synchronization       & uavcan.protocol.GlobalTimeSync &  & Needed by publishers of timestamped data \\
    Node configuration         & uavcan.protocol.param.GetSet uavcan.protocol.param.ExecuteOpcode &  &  \\
    Node restart               & uavcan.protocol.RestartNode &  & Recommended, may be used to refresh node settings \\
    File transfer              & uavcan.protocol.file.GetInfo uavcan.protocol.file.GetDirectoryEntryInfo
                                uavcan.protocol.file.Delete uavcan.protocol.file.Read
                                uavcan.protocol.file.Write uavcan.protocol.file.EntryType
                                uavcan.protocol.file.Error uavcan.protocol.file.Path &  &  \\
    Firmware update            & uavcan.protocol.file.BeginFirmwareUpdate &  & Requires \emph{File transfer} \\
    Debug features             & uavcan.protocol.debug.KeyValue uavcan.protocol.debug.LogMessage
                                uavcan.protocol.debug.LogLevel  &  & Lowest-priority messages \\
    Command shell access       & uavcan.protocol.AccessCommandShell  &  &  \\
    Panic mode                 & uavcan.protocol.Panic &  &  \\
    Dynamic node ID allocation & uavcan.protocol.dynamic\_node\_id.Allocation  &  & Uses anonymous transfers \\
    \bottomrule
\end{tabular}
\caption{UAVCAN protocol: high-level functionalities}\label{tab:uc_func}
\end{table}

As part of the \emph{Node configuration} function, the following parameters should be available by default:
\begin{itemize}
    \item Data type ID
    \item Message publication period
    \item Transfer priority
    \item Node ID
    \item CAN bus bit rate
    \item Instance ID
\end{itemize}


\subsubsection{Library}
UAVCAN provides a software stack written in C++ called \texttt{libuavcan}.
It consists in a portable, cross-platform library, a package of standard message definitions written in a custom data definition language, and a code generator that converts these into header files for application-level inclusion.

Libuavcan depends on only a very limited portion of the standard C++ library, making it compilable on most moder compilers and embeddable in microcontroller architectures and full-fledged systems alike.
Through compile-time configuration, it gives the possibility of ruling out processing or memory-costly or parts, such as exception handling, run-time type identification (RTTI), and dynamic memory allocation.
The latter is a desirable feature on many embedded and real-time system.

A major characteristic of libuavcan is that it offers built-in support for all of the high-level functions presented above.
These can be individually added to a project by including the relevant header files from \texttt{uavcan/protocol/} and initializing the service providers.
Although the recommended/suggested functions are included by default, the developer has the possibility of disabling them through the \texttt{UAVCAN\_TINY} variable, in order to further reduce the memory footprint.
This is however only recommended for deeply embedded systems with severe memory constraints, as it will also remove support for transmission diagnostics, event logging, and memory pool allocation.

For the purpose of this project, the library has been configured to compile with \texttt{C++11} support and all suggested functions available.
All other settings have been kept to their default value.

Depending on the development environment used, the drivers and library can be built using \texttt{cmake}, \texttt{make}, or any IDE-specific build system, as long as the necessary source files are present and the libuavcan \text{include} folder added to the inclusion path.


\paragraph{Platform driver}
The library is not bounded to any specific CAN interface, and relies on a platform-dependent driver.
As of today, the officially supported platforms are:
\begin{itemize}
    \item \emph{Linux} (requires \texttt{libposix4} and \texttt{SocketCAN}), most comprehensive driver
    \item \emph{POSIX}, only abstract interface
    \item \emph{LPC11C24}, minimal implementation, no multi-threading
    \item \emph{STM32}, middle-ground support
\end{itemize}

The drivers are responsible for implementing the low-level, platform-dependent functions, which are abstracted by the following C++ classes:
\begin{itemize}
    \item \texttt{uavcan::ISystemClock}, providing monotonic clock (invariant) and network clock (synchronized and adjusted across the nodes)
    \item \texttt{uavcan::ICanIface}, providing \texttt{send} and \texttt{receive} functions, as well as CAN filters configuration
    \item \texttt{uavcan::ICanDriver}, which acts as a container for supporting multiple redundant interfaces.
\end{itemize}

The libuavcan driver for STM32 has no third-party dependencies (except if RTOS support is enabled).
It supports all of the major capabilities, including redundant CAN interfaces (whenever available on the MCU), clock synchronization (through general-purpose timer), and has bindings for several RTOS (ChibiOS, NuttX, and FreeRTOS).

The driver too is configurable via preprocessor definitions.
The ones used in the project are reported below.

\begin{table}[H]
\centerfloat
\begin{tabular}{@{} >{\ttfamily}m{10em} m{5em} m{15em} @{}}
    \toprule
    \normalfont{Parameter}              & \normalfont{Value}    & Description \\
    \midrule
    UAVCAN\_STM32\_CHIBIOS              & 1     & Symbol defining the platform used; mandatory \\
    UAVCAN\_STM32\_TIMER\_NUMBER        & 7     & Specifies hardware time used for clock; mandatory \\
    UAVCAN\_STM32\_NUM\_IFACES          & 1     & Number of available CAN interfaces \\
    UAVCAN\_STM32\_IRQ\_PRIORITY\_MASK  & 4     & Priority level of driver-related interrupts \\
    \bottomrule
\end{tabular}
\caption{UAVCAN driver: compile-time settings}\label{tab:uc_drv_sett}
\end{table}
