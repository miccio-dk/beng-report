% !TEX root = ../root.tex
\chapter{Conclusions}\label{ch:concl}


\section{Product assessment}

Conclusions based on the product, e.g. if it fulfilled all the requirements, if it yielded the desired performances etc \dots
The project did not manage to reach a satisfactory completion level.
As of now, the product doesn't fulfill its purpose and must undergo heavy development before it can be used as a proof of concept, let alone considered complete.
Indeed, most its implementation exists only on a theoretical basis.

However, the compilation of this document contributed in shedding some light on the encountered issue, and has allowed the formalization of the future implementation steps.

On a purely practical standpoint, the set of tools and reusable software modules implemented during the course of the project will likely reveal useful in the future, and add value to the company.

Despite the amount of time lost while migrating from one architecture to the other, the possibility of experimenting with a different platform --- in particular a more powerful and extensively supported one --- will benefit the future development of the product as a whole.

For what concerns the decaWave modules, their stability is still far from behind even adequate.
Although the newer PCB layout has been a mayor lead in usability, consecutive acquisition of ranges fails a staggering three fourth of the times, hinting that some issues are still present.

% !TEX root = ../root.tex
\section{Process assessment}\label{sec:concl_proc}

Conclusions based on the development, e.g. how lean it has been, any useful tool discovered, any cumbersome aspect of the workflow that coud have been improved \dots

use risk management to explain escalation of issues:
e' stato veritiero?
e' servito a un cazzo?

The software development and integration process and suffered from several sources of hindrance.

In particular, commitment to previously-established projects has played a large role in shifting most of the focus and productivity away from this very project and towards other endeavors considered more imperative by the company.
However, the choice of not employing the necessary firmness is one for which I must take full responsibility.

Another major setback has been the delayed delivery of the second iteration of printed circuit boards (along with several other designs) which has left anyone in distress and mainly goes to prove how a proper risk management plan can really make a difference.

On a side note, the chance of managing several complex repositories on a on-premises infrastructure has been particularly useful and will certainly be beneficial for future projects.

% !TEX root = ../root.tex
\section{Possible improvements}\label{sec:concl_fut}

Suggestions on improvements based on missing features, company and supervisors feedbacks, etc \dots

As both UAVComponents and myself have put a considerable amount of resources for what is nevertheless a project with possible interesting applications, the development is most likely to resume.

With the improved programming skills and the acquisition of a broad corpus of knowledge related to ultra-wideband, CAN, and embedded development practices, the project and process alike will hopefully experience a more focused and lean management.

As for possible improvements and features (besides those yet to be implemented), the idea of employing some form of data filtering may substantially improve the real-time behavior of the localization system, which would otherwise be quite limited.
