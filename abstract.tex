\begin{abstract}

This document was composed by Riccardo Miccini (s137345) as part of a 15 ECTS-point project in fulfillment of the requirements for granting a B.Eng. (Bachelor of Engineering) in Electronics and Computer Engineering from the Center for Bachelor of Engineering Studies of the Technical University of Denmark.

This document covers the development of an indoor positioning system product to be deployed in autonomous vehicles and the like.
In particular, great focus will be given to the integration and explanation of \emph{UAVCAN}, an emerging technology in the field of UAV integration.
The project has been carried out in collaboration with UAVComponents ApS.

The content of this document is divided using the following scheme: an introduction to the project and its context, an analysis modeled upon elements of the Unified Process, a comprehensive description of the implementation details of the necessary hardware and software, inclusive of description of third-party components, and a testing chapter.
Conclusions will be drawn, assessing the overall results and the newly gained experience and skills.

Relevant support material is supplied along with this document in digital form, and its content is described in the appendix.

Due to unforeseen difficulties, most of the implementation details discussed over the course of these chapters have not been carried out.
Nevertheless, the document will strive to provide the necessary knowledge for evaluating the extent of the project.

\bigskip

\emph{The opinions expressed in this document are the author's own and do not reflect the view of UAVComponents and its personnel.}

\emph{All brands, product names, logos, or other trademarks featured or referred to in this document are the property of their respective holders, and their use does not imply endorsement.}

\end{abstract}
